\documentclass[a4paper,12pt]{article}

\usepackage[margin=90pt]{geometry}
\usepackage[T1,T2A]{fontenc}
\usepackage[utf8]{inputenc}
\usepackage[maxbibnames=99]{biblatex}
\usepackage[bulgarian]{babel}
\usepackage[unicode]{hyperref}
\usepackage{enumitem}
\usepackage{upgreek}
\usepackage{url}
\usepackage{graphicx}
\usepackage{mathtools}
\usepackage{indentfirst}
\usepackage[autostyle=false]{csquotes}
\usepackage{textcomp, gensymb}
\usepackage{subcaption}
\usepackage{bm,amsmath,amsfonts,amssymb}
\usepackage{listings}
\usepackage{xcolor}
\usepackage{tabularx}
\usepackage{amssymb}

\definecolor{codegreen}{rgb}{0,0.6,0}
\definecolor{codegray}{rgb}{0.5,0.5,0.5}
\definecolor{codepurple}{rgb}{0.58,0,0.82}
\definecolor{backcolour}{rgb}{0.95,0.95,0.92}

\lstdefinestyle{mystyle}{
    backgroundcolor=\color{backcolour},   
    commentstyle=\color{codegreen},
    keywordstyle=\color{magenta},
    numberstyle=\tiny\color{codegray},
    stringstyle=\color{codepurple},
    basicstyle=\ttfamily\footnotesize,
    breakatwhitespace=false,         
    breaklines=true,                 
    captionpos=b,                    
    keepspaces=true,                 
    numbers=left,                    
    numbersep=5pt,                  
    showspaces=false,                
    showstringspaces=false,
    showtabs=false,                  
    tabsize=2,
}

\lstset{style=mystyle}

\addbibresource{./references.bib}
\graphicspath{ {./img/} }

\DeclareUnicodeCharacter{0301}{\'{e}}


\DeclareCaptionFormat{custom}
{%
    \textbf{\small #1#2}\textit{\small #3}
}
\captionsetup{format=custom}

\begin{document}
\title{
    Дипломна работа\\
    \large Търсене на исторически изображения по описание}
    \author{
        Иван Христов, ФН - 2MI3400066\\
        Специалност - Информатика, Магистърска програма - Изкуствен интелект\\
        Научен ръководител - проф. д-р Иван Койчев\\
        Консултант - доц. д-р Милена Добрева
    }

\maketitle

\begin{figure}[h]
    \centering
    \includegraphics[width=250px]{fmi.png}
    \label{fig:fmi}
\end{figure}

\pagebreak

\tableofcontents
\pagebreak

% =============================================================================================================
\section{Увод}

% *************************************************************************************************************
\subsection{Актуалност на проблема и мотивация}

Музеите са богати на информация, свързана с разнородни исторически личности, предмети от бита, съкровища, летописи и други. Носителите на тази информация са предимно материални и разнородни. Такива са писания, хартиени снимки, оръжия, дрехи и други. Изключения са видео и аудио записи, които запечтват моменти от близкото минало. Впрочем един музей би могъл да се разглежда като база от данни, която може да се дигитализира. Това е лесно постижимо чрез правенето на снимки.

\bigbreak

Всъщност голяма част от информацията вече е налична под формата на изображения, но тя често е неструктурирана, хаотична и без необходимите за нея описания. В резултат, извличането на данни от музей е бавно и трудоемко. Критериите за търсене се подават на естествен език и те се обработват от хора историци. Обичайно това е единственият начин за извличане на информация. Така търсенето е бавно с възможност за допускане на грешки и пропуск на информация. Не бива да се пренебрегват времето и материалните ресурси, вложени в обучението на историци.

\bigbreak

Троянският музей е пример за институция с проблеми от това естество. Той специализира в съвременна история на България, в ч. възрожденска епоха и царско време. Музеят разполага с богата колекция от народни носии, предмети от бита, дърворезби, летописи и прочие. Дигитална информация за част от нея, е налична под формата на множество снимки - около \textit{4TB} в \textit{JPG} формат. Някой от тях са заснети чрез фотоапарат, докато други са сканирани. Съответно те са подложени на различни афинни трансформации (ротация, скалиране, отместване и др.), дисторция и илюминация. Пример за това е, че размерите на изображенията варират от около 200KB до около 10MB. Детайлността на изображенията варира значително - някои от тях са с тълпи от хора, докато други са с малко на брой субекти/обекти. Това би могло да
бъде проблем при наличие на ниска резолюция на изображенията. Снимките от музея са исторически, съответно има наличие на черно-бели и цветни
изображения.

\bigbreak

Архивът на Троянският музей със своето разнообразие е достатъчно пред-
ставителен за проблема, но той далеч не е единственият такъв. Пример за
подобно множество от снимки е това на Народната библиотека в Пловдив.
Следователно бързото и надеждно извличане на исторически изображения е проблем, чиято автоматизация би имала повсеместен културен и исторически принос.

% *************************************************************************************************************
\subsection{Цел и задачи на дипломната работа}

Настоящата дипломна работа се концентрира над изграждането на система за предлагане на изображения, чието съдържание отговаря на описание от натурален текст. Допускаме, че изображенията не са подредени, както и че някои нямат съпътстваща информация (напр. надписи). Примерен сценарий на употреба е - потребител търси “зелено поле с течащи чешми”, при което системата връща изображения с чешми и зелени полета. В литературата тази задача е позната като \textbf{"семантично търсене на изображения"}.

\bigbreak

Като част от задачата се разглежда интеграцията на системата с данни от български музеи (в ч. Троянския музей). Много от изображенията в музеите нямат надписи. Нови експонати биват редовно добавяни. Съответно за разлика от съществуващи решения на задачата за семантично търсене, тази дипломна работа цели поддръжката на \textit{online} обучение и търсене на изображения без надписи. Системата трябва коректно да обработва именувани същности (\textit{named entities}) и стари български думи.

\bigbreak

Процесът по решаване на задачата е разделен на няколко части. Предварителна подготовка на база от данни - (1) подбор на обучително множество от данни с предварително надписани изображения, (2) подготовка (в ч. надписване) на подмножество от изображенията от Троянския музей, (3) избор/подготовка на алгоритми за кодиране на надписи и матрици на изображения, (4) изграждане на индекси за бързо търсене на изображения, съответо по сходни надписи и матрици. Търсене на надписани изображения по текст - (1) кодиране на текст, (2) използване на индекс за бързо търсене на сходни по надписи изображения. Търсене на надписани и ненадписани изображения - (1) кодиране на текст, (2) използване на индекс за бързо търсене на сходни по надписи изображения, (3) кодиране на матриците на сходните надписани изображения, (4) използване на индекс за обратно търсене (\textit{reverse search}) на сходни изображения чрез кодовете на сходните надписани изображения.

% *************************************************************************************************************
\subsection{Очаквани ползи от реализацията}

Наличието на система за семантично търсене на изображения би изместило фокуса на историците от ежедневни протоколни задачи към въпроси, изискващи повече ментален капацитет. Впрочем решението на задачата на дипломната работа трябва да се разглежда не като заместник, а като помощник на историците.

\bigbreak

Като страничен ефект от търсенето на ненадписани изображения се осъществява откриването на знания. Възможно е да бъдат открити непознати досега асоциации между изображения (напр. шарките на една народна носия да приличат на шарките на друга). Една такава скрита асоциация е дублиране между изображения с отклонения откъм афинни и цветови трансформации, перспектива и други. Чрез системата, откриването и отстраняването на дубликати би се улеснило.

\bigbreak

Съществуват изображения, които не са надписани, но съдържат надписи. Такива са снимки на текстове. Системата за семантично търсене позволява интеграция на решения за разпознаване на букви (\textit{character recognition}), с цел ускорение търсенето на такива документи.

% *************************************************************************************************************
\subsection{Структура на дипломната работа}

% =============================================================================================================
\section{Преглед на предметната област}

% *************************************************************************************************************
\subsection{Основни дефиниции}

В тази секция се дефинира задачата за \textbf{търсене на изображения по описание}. В контекста на задачата ще използваме думите \textbf{описание} и \textbf{надпис} взаимозаменяемо. Нека имаме множество от изображения $I$, където за всяко изображение $i \in I$ съществува множество от \textbf{описания}. Описание $c$ представяме чрез \textbf{естествен език}. Изображение $i$ предаставяме чрез правоъгълна матрица $m$, такава че $m \in \mathbb{N}^{n \times m}, n,m \in \mathbb{N}$.

\bigbreak

Нека $c_1$, $c_2$ и $c_3$ са описания, където $c_1$ е по-близко по смисъл до $c_2$ отколкото до $c_3$. \textbf{Кодираща функция на описание} наричаме сюрекция $\lambda$, такава че $\lambda(c) \in \mathbb{R}^n$, където $n \in \mathbb{N}$ и $\| \lambda(c_1) - \lambda(c_2) \| <= \| \lambda(c_1) - \lambda(c_3) \|$. Като пример може да се разгледат $c_1 = $ `Бяла котка и черно куче си играят', $c_2 = $ `Котка скача над куче' и $c_3 = $ `Камила пие вода в оазис'. В останалата част от дипломната ще използваме $\lambda$ за обозначение на кодираща функция на описание.

\bigbreak

Преди да дефинираме кодираща функция на изображение, трябва дадем няколко статистически дефиниции. \textbf{Случаен експеримент} наричаме експеримент, при който условията не определят еднозначно резултата. \textbf{Основно пространство} наричаме множеството от всички възможни изходи на случаен експеримент. Бележи се чрез $\Omega$ и има поне $2$ елемнта.

\bigbreak

\textbf{Случайна величина} наричаме величина, чиято стойност зависи от резултата на случаен експеримент. Обща дефиниция е:

\[\xi : \begin{cases}
    \Omega \rightarrow \mathbb{R}\\
    \omega \rightarrow \xi(\omega)
\end{cases} \], т.е. $\xi(\omega)$ е реално число.

\bigbreak

\textbf{Закон за разпределение} наричаме съотвествие между стойностите на случайна величина и вероятностите на тези стойности

\begin{tabular}{|c|c|c|c|c|}
    \hline
    $\xi$ & $x_1$ & $\dots$ & $x_n$ & $\dots$ \\
    \hline
    $P$ & $p_1$ & $\dots$ & $p_n$ & $\dots$ \\
    \hline
\end{tabular}

където числата $x_1,\dots,x_n$ са възможните стойности на сл. величина, а числата 
$p_i=P(\{\xi = x_i\}), 0 \leq p_i \leq 1, i=1,\dots,n,\dots$ са вероятностите, с които $\xi$ приема тези стойнсти.

\bigbreak

\textbf{Метрично пространство} наричаме наредената двойка $(M, d)$, където $M$ е множество и $d : M \times M \to \mathbb{R}$ е такава, че:

\begin{enumerate}
    \item $d(x, x) = 0$
    \item ако $x \neq y$ и $x, y \in M$, то $d(x, y) > 0$
    \item е симетрична, т.е. $d(x, y) = d(y, x)$
    \item е спазено неравенството на триъгълника, т.е. $d(x, y) <= d(x, z) + d(z, y)$, където $x,y,z \in M$.
\end{enumerate}

Функцията $d$ наричаме \textbf{метрика}.

\bigbreak

\textbf{Статистическа дистанция} оразмерява дистанцията между два статистически обекта. Статистическите обекти може да са две случайни величини, две разпределения или други. Статистическата дистанция може да бъде метрика. Някои често използвани метрики в практиката са вариационната дистанция, метриките Васерщайн и Хелингер и други.

\bigbreak

Нека $i_1$, $i_2$, $i_3$ са изображения, където $i_1$ е по-близко по смисъл до $i_2$ отколкото до $i_3$. Нека $\phi$ е сюрекция, такава че $\phi(i) \in \mathbb{R}^{n \times m}, n,m \in \mathbb{N}$. По-просто казано за всяко изображение връща $m$ на брой реални вектора. Всеки от тези реални вектори може да се интерпретира като статистическо разпределение с $n$ възможни стойности на случайната величина. Нека $d$ е метрика за статистическа дистанция в метричното пространство $(\mathbb{R}^{n \times m}, d)$. Наричаме $\phi$ \textbf{кодираща функция на изображение} точно тогава когато $d(\phi(i_1), \phi(i_2)) \leq d(\phi(i_1), \phi(i_3))$.

\bigbreak

Класическите системи за търсене на изображения по описание разчитат на детерминистични подходи за решаване на задачата. Пример за примитивно решение е търсене по ключови думи, където системата извежда само изображения, чиито описания еднозначно съдържат въведените ключови думи. Очевидно подобрение на този подход е да се използват \textbf{стемиране} и \textbf{токенизация}. Стемиране наричаме извличането на корена на дадена дума. Токенизация наричаме разбиването на низ до по-малки смислени парчета наречени \textbf{токени} (напр. изречения, думи, срички и символи). Тези похвати позволяват търсене по цели изречения, където системата извежда изображения, чиито описания съдържат някои от токените на входното описание. При търсене по цели изображения е възможно потребителят да въведе \textbf{стоп думи}, които биха могли да имат негативно отражение върху точността на търсене. Стоп думи наричаме тези, които често биват употребявани в даден език, но не носят особена семантика. При търсене е редно те да се отстраняват от входните изречения.

\bigbreak

Колкото и да са прагматични и полезни, класическите системи не позволяват търсене по семантика. Нека разгледаме пример, където системата знае само за изображение с надпис 'къртица`. При търсене по описание 'незрящо копаещо животно`, класическа система, не би върнала изображението. С такива заявки биха могли да се справят системи за семантично търсене.

\bigbreak

Трябва да правим разлика между семантика на изображение и семантика на описание. 

\bigbreak

в практиката има изображения с липсващи надписи и такива със зададени надписи, да изобразя връзката между описания и семантика в самото изречение

В литературата се среща под името семантично търсене на изображения (\textit{semantic image search}). 

\bigbreak

Задачата за семантично търсене на изображения е специализация на тази за семантично търсене на документи. 

Кодиране/Декодиране

Кодиране на изображения -> SIFT, 

Семантично търсене на изображения

% *************************************************************************************************************
\subsection{Подходи за решаване на проблемите}

% *************************************************************************************************************
\subsection{Съществуващи решения (практически реализации)}

% *************************************************************************************************************
\subsection{Избор на критерии за сравнение и сравнителен анализ на решения}

% *************************************************************************************************************
\subsection{Изводи}

% =============================================================================================================
\printbibliography[title={Използвана литература}]

\end{document}
