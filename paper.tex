\documentclass[a4paper,12pt]{article}

\usepackage[margin=90pt]{geometry}
\usepackage[T1,T2A]{fontenc}
\usepackage[utf8]{inputenc}
\usepackage[
backend=biber,
style=ieee
]{biblatex}
\usepackage[bulgarian]{babel}
\usepackage[unicode]{hyperref}
\usepackage{enumitem}
\usepackage{upgreek}
\usepackage{url}
\usepackage{graphicx}
\usepackage{mathtools}
\usepackage{indentfirst}
\usepackage[autostyle=false]{csquotes}
\usepackage{textcomp, gensymb}
\usepackage{subcaption}
\usepackage{bm,amsmath,amsfonts,amssymb}
\usepackage{listings}
\usepackage{xcolor}
\usepackage{tabularx}
\usepackage{amssymb}

\definecolor{codegreen}{rgb}{0,0.6,0}
\definecolor{codegray}{rgb}{0.5,0.5,0.5}
\definecolor{codepurple}{rgb}{0.58,0,0.82}
\definecolor{backcolour}{rgb}{0.95,0.95,0.92}

\lstdefinestyle{mystyle}{
    backgroundcolor=\color{backcolour},   
    commentstyle=\color{codegreen},
    keywordstyle=\color{magenta},
    numberstyle=\tiny\color{codegray},
    stringstyle=\color{codepurple},
    basicstyle=\ttfamily\footnotesize,
    breakatwhitespace=false,         
    breaklines=true,                 
    captionpos=b,                    
    keepspaces=true,                 
    numbers=left,                    
    numbersep=5pt,                  
    showspaces=false,                
    showstringspaces=false,
    showtabs=false,                  
    tabsize=2,
}

\lstset{style=mystyle}

\addbibresource{./references.bib}
\graphicspath{ {./img/} }

\DeclareUnicodeCharacter{0301}{\'{e}}


\DeclareCaptionFormat{custom}
{%
    \textbf{\small #1#2}\textit{\small #3}
}
\captionsetup{format=custom}

\begin{document}
\title{
    Дипломна работа\\
    \large Търсене на исторически изображения по описание}
    \author{
        Иван Христов, ФН - 2MI3400066\\
        Специалност - Информатика, Магистърска програма - Изкуствен интелект\\
        Научен ръководител - проф. д-р Иван Койчев\\
        Консултант - доц. д-р Милена Добрева
    }

\maketitle

\begin{figure}[h]
    \centering
    \includegraphics[width=250px]{fmi.png}
    \label{fig:fmi}
\end{figure}

\pagebreak

\tableofcontents
\pagebreak

% =============================================================================================================
\section{Увод}

% *************************************************************************************************************
\subsection{Актуалност на проблема и мотивация}

Музеите са богати на информация, свързана с разнородни исторически личности, предмети от бита, съкровища, летописи и други. Носителите на тази информация са предимно материални и разнородни. Такива са писания, хартиени снимки, оръжия, дрехи и други. Изключения са видео и аудио записи, които запечтват моменти от близкото минало. Впрочем един музей би могъл да се разглежда като база от данни, която може да се дигитализира. Това е лесно постижимо чрез правенето на снимки.

\bigbreak

Всъщност голяма част от информацията вече е налична под формата на изображения, но тя често е неструктурирана, хаотична и без необходимите за нея описания. В резултат, извличането на данни от музей е бавно и трудоемко. Критериите за търсене се подават на естествен език и те се обработват от хора историци. Обичайно това е единственият начин за извличане на информация. Така търсенето е бавно с възможност за допускане на грешки и пропуск на информация. Не бива да се пренебрегват времето и материалните ресурси, вложени в обучението на историци.

\bigbreak

Троянският музей е пример за институция с проблеми от това естество. Той специализира в съвременна история на България, в ч. възрожденска епоха и царско време. Музеят разполага с богата колекция от народни носии, предмети от бита, дърворезби, летописи и прочие. Дигитална информация за част от нея, е налична под формата на множество снимки - около \textit{4TB} в \textit{JPG} формат. Някой от тях са заснети чрез фотоапарат, докато други са сканирани. Съответно те са подложени на различни афинни трансформации (ротация, скалиране, отместване и др.), дисторция и илюминация. Пример за това е, че размерите на изображенията варират от около 200KB до около 10MB. Детайлността на изображенията варира значително - някои от тях са с тълпи от хора, докато други са с малко на брой субекти/обекти. Това би могло да
бъде проблем при наличие на ниска резолюция на изображенията. Снимките от музея са исторически, съответно има наличие на черно-бели и цветни
изображения.

\bigbreak

Архивът на Троянският музей със своето разнообразие е достатъчно пред-
ставителен за проблема, но той далеч не е единственият такъв. Пример за
подобно множество от снимки е това на Народната библиотека в Пловдив.
Следователно бързото и надеждно извличане на исторически изображения е проблем, чиято автоматизация би имала повсеместен културен и исторически принос.

% *************************************************************************************************************
\subsection{Цел и задачи на дипломната работа}

Настоящата дипломна работа се концентрира над изграждането на система за предлагане на изображения, чието съдържание отговаря на описание от натурален текст. Допускаме, че изображенията не са подредени, както и че някои нямат съпътстваща информация (напр. надписи). Примерен сценарий на употреба е - потребител търси “зелено поле с течащи чешми”, при което системата връща изображения с чешми и зелени полета. В литературата тази задача е позната като \textbf{"семантично търсене на изображения"}.

\bigbreak

Като част от задачата се разглежда интеграцията на системата с данни от български музеи (в ч. Троянския музей). Много от изображенията в музеите нямат надписи. Нови експонати биват редовно добавяни. Съответно за разлика от съществуващи решения на задачата за семантично търсене, тази дипломна работа цели поддръжката на \textit{online} обучение и търсене на изображения без надписи. Системата трябва коректно да обработва именувани същности (\textit{named entities}) и стари български думи.

\bigbreak

Процесът по решаване на задачата е разделен на няколко части. Предварителна подготовка на база от данни - (1) подбор на обучително множество от данни с предварително надписани изображения, (2) подготовка (в ч. надписване) на подмножество от изображенията от Троянския музей, (3) избор/подготовка на алгоритми за кодиране на надписи и матрици на изображения, (4) изграждане на индекси за бързо търсене на изображения, съответо по сходни надписи и матрици. Търсене на надписани изображения по текст - (1) кодиране на текст, (2) използване на индекс за бързо търсене на сходни по надписи изображения. Търсене на надписани и ненадписани изображения - (1) кодиране на текст, (2) използване на индекс за бързо търсене на сходни по надписи изображения, (3) кодиране на матриците на сходните надписани изображения, (4) използване на индекс за обратно търсене (\textit{reverse search}) на сходни изображения чрез кодовете на сходните надписани изображения.

% *************************************************************************************************************
\subsection{Очаквани ползи от реализацията}

Наличието на система за семантично търсене на изображения би изместило фокуса на историците от ежедневни протоколни задачи към въпроси, изискващи повече ментален капацитет. Впрочем решението на задачата на дипломната работа трябва да се разглежда не като заместник, а като помощник на историците.

\bigbreak

Като страничен ефект от търсенето на ненадписани изображения се осъществява откриването на знания. Възможно е да бъдат открити непознати досега асоциации между изображения (напр. шарките на една народна носия да приличат на шарките на друга). Една такава скрита асоциация е дублиране между изображения с отклонения откъм афинни и цветови трансформации, перспектива и други. Чрез системата, откриването и отстраняването на дубликати би се улеснило.

\bigbreak

Съществуват изображения, които не са надписани, но съдържат надписи. Такива са снимки на текстове. Системата за семантично търсене позволява интеграция на решения за разпознаване на букви (\textit{character recognition}), с цел ускорение търсенето на такива документи.

% *************************************************************************************************************
\subsection{Структура на дипломната работа}

% =============================================================================================================
\section{Обзор на предметната област}

% *************************************************************************************************************
\subsection{Основни понятия}

В тази секция се дефинира задачата за \textbf{търсене на изображения по описание}. В контекста на дипломната работа ще използваме думите \textbf{описание} и \textbf{надпис} взаимозаменяемо. Задачата за търсене на изображения по описание е генерализация на тази за \textbf{търсене на документи по критерий}, където изображенията са документите, а описанията са критериите. Нека имаме множество от изображения $I$, където за всяко изображение $i \in I$ съществува множество от \textbf{описания}. Описание $c$ представяме чрез \textbf{естествен език}. Изображение $i$ предаставяме чрез правоъгълна матрица $m$, такава че $m \in \mathbb{N}^{n \times m}, n,m \in \mathbb{N}$.

\bigbreak

Нека $c_1$, $c_2$ и $c_3$ са описания, където $c_1$ е по-близко по смисъл до $c_2$ отколкото до $c_3$. \textbf{Кодираща функция на описание} наричаме сюрекция $\lambda$, такава че $\lambda(c) \in \mathbb{R}^n$, където $n \in \mathbb{N}$ и $\| \lambda(c_1) - \lambda(c_2) \| <= \| \lambda(c_1) - \lambda(c_3) \|$. \textbf{Код на описание} $c$ наричаме $\lambda(c)$. Като пример може да се разгледат $c_1 = $ `Бяла котка и черно куче си играят', $c_2 = $ `Котка скача над куче' и $c_3 = $ `Камила пие вода в оазис'. В останалата част от дипломната работа ще използваме $\lambda$ за обозначение на кодираща функция на описание.

\bigbreak

Преди да дефинираме кодираща функция на изображение, трябва дадем няколко статистически дефиниции. \textbf{Случаен експеримент} наричаме експеримент, при който условията не определят еднозначно резултата. \textbf{Основно пространство} наричаме множеството от всички възможни изходи на случаен експеримент. Бележи се чрез $\Omega$ и има поне $2$ елемнта.

\bigbreak

\textbf{Случайна величина} наричаме величина, чиято стойност зависи от резултата на случаен експеримент. Обща дефиниция е:

\[\xi : \begin{cases}
    \Omega \rightarrow \mathbb{R}\\
    \omega \rightarrow \xi(\omega)
\end{cases} \], т.е. $\xi(\omega)$ е реално число.

\bigbreak

\textbf{Закон за разпределение} наричаме съотвествие между стойностите на случайна величина и вероятностите на тези стойности

\begin{tabular}{|c|c|c|c|c|}
    \hline
    $\xi$ & $x_1$ & $\dots$ & $x_n$ & $\dots$ \\
    \hline
    $P$ & $p_1$ & $\dots$ & $p_n$ & $\dots$ \\
    \hline
\end{tabular}

където числата $x_1,\dots,x_n$ са възможните стойности на сл. величина, а числата 
$p_i=P(\{\xi = x_i\}), 0 \leq p_i \leq 1, i=1,\dots,n,\dots$ са вероятностите, с които $\xi$ приема тези стойнсти.

\bigbreak

\textbf{Метрично пространство} наричаме наредената двойка $(M, d)$, където $M$ е множество и $d : M \times M \to \mathbb{R}$ е такава, че:

\begin{enumerate}
    \item $d(x, x) = 0$
    \item ако $x \neq y$ и $x, y \in M$, то $d(x, y) > 0$
    \item е симетрична, т.е. $d(x, y) = d(y, x)$
    \item е спазено неравенството на триъгълника, т.е. $d(x, y) <= d(x, z) + d(z, y)$, където $x,y,z \in M$.
\end{enumerate}

Функцията $d$ наричаме \textbf{метрика}.

\bigbreak

\textbf{Статистическа дистанция} оразмерява дистанцията между два статистически обекта. Статистическите обекти може да са две случайни величини, две разпределения или други. Статистическата дистанция може да бъде метрика. Някои често използвани метрики в практиката са вариационната дистанция, метриките Васерщайн и Хелингер и други.

\bigbreak

Нека $i_1$, $i_2$, $i_3$ са изображения, където $i_1$ е по-близко по смисъл до $i_2$ отколкото до $i_3$. Нека $\phi$ е сюрекция, такава че $\phi(i) \in \mathbb{R}^{n \times m}, n,m \in \mathbb{N}$. По-просто казано за всяко изображение връща $m$ на брой реални вектора. Всеки от тези реални вектори може да се интерпретира като статистическо разпределение с $n$ възможни стойности на случайната величина. Нека $d$ е метрика за статистическа дистанция в метричното пространство $(\mathbb{R}^{n \times m}, d)$. Наричаме $\phi$ \textbf{кодираща функция на изображение} точно тогава когато $d(\phi(i_1), \phi(i_2)) \leq d(\phi(i_1), \phi(i_3))$. \textbf{Код на изображение} $i$ наричаме $\phi(i). $ В останалата част от дипломната работа ще използваме $\phi$ за обозначение на кодираща функция на описание.

\bigbreak

Традиционно системите за търсене на изображения по описание разчитат на детерминистични подходи за решаване на задачата. Пример за примитивно решение е търсене по ключови думи, където системата извежда само изображения, чиито описания еднозначно съдържат въведените ключови думи. Очевидно подобрение на този подход е да се използват \textbf{стемиране} и \textbf{токенизация}. Стемиране наричаме извличането на корена на дадена дума. Токенизация наричаме разбиването на низ до по-малки смислени парчета наречени \textbf{токени} (напр. изречения, думи, срички и символи). Тези похвати позволяват търсене по цели изречения, където системата извежда изображения, чиито описания съдържат някои от токените на входното описание. При търсене по цели изображения е възможно потребителят да въведе \textbf{стоп думи}, които биха могли да имат негативно отражение върху точността на търсене. Стоп думи наричаме тези, които често биват употребявани в даден език, но не носят особена семантика. При търсене е редно те да се отстраняват от входните изречения.

\bigbreak

Въпреки, че са прагматични и полезни, традиционните системи за търсене на изображения по описание не позволяват търсене по семантика. Нека разгледаме пример, където системата знае само за изображение с надпис 'къртица`. При търсене по описание 'незрящо копаещо животно`, класическа система, не би върнала изображението. С такива заявки биха могли да се справят системи за семантично търсене.

\bigbreak

\textbf{Семантично търсене на документи} наричаме подход за решаване на задачата за търсене на документи по критерий, който се опитва да подобри точността на търсене чрез разбиране на значението на критерия. Обичайно това се постига чрез комбинация от енкодер и декодер, съответно към и от векторно пространство $V$. Важно е да отбележим, че съществуват множество други подходи, сред които и такива базирани на онтологии и логическо програмиране. За целите на дипломната работа нека с $C$ означим множеството от всички възможни критерии и с $D$ означим множеството от всички документи. Очевидно $D$ е крайно изброимо множество. \textbf{Енкодер} ще наричаме сюрекция $f$, където $f: C \to V$. \textbf{Декодер} ще наричаме сюрекция $g$, където $g: V \to S^k$, където $k \in \mathbb{N}$ е броят на върнатите документи и $S = D \times \mathbb{R}$ наричаме множеството от всички оценени по релевантност документи.

\bigbreak

Съществуват два вида семантично търсене на документи - симетрично и асиметрично. Едно търсене е \textbf{симетрично} когато елементите на $C$ са със същата размерност и тип като тези на $D$. Ако едно семантично търсене не е симетрично, то е \textbf{асиметрично}. Примери за симетрично са търсене на описание по сходно описание или търсене на изображение по сходно изображение (така нареченото обратно търсене на изображения (\textit{reverse image search})). Примери за асиметрично са търсене на изображение по описание или книга по описание.

\bigbreak

Пример за енкодер във векторно пространство $\mathbb{R}^n$ е кодиращата функция на описания $\lambda$. Аналогично, кодиращата функция на изображения $\phi$ е пример за енкодер във векторно пространство $\mathbb{R}^{n \times m}$. Пример за декодер е функция, която знаейки кодовете на изображенията, намира оценка за релевантност на изображенията и връща първите няколко с най-висока такава. Оценка за релевантност може да бъде дадена чрез някоя от споменатите статистически метрики (Васерщайн, Хелингер и др.). Често използвана е \textit{tf-idf}(\textit{term frequency-inverse document frequency}).

\bigbreak

\textbf{TF-IDF} е оценка за релевантност на дума към документ от корпус. Корпус е множество от документи. Ще използваме терм вместо дума за повече интуитивност. По-формално нека $tf(t, d) = \frac{f_{t,d}}{\sum_{t' \in d} f_{t', d}}$, където $f_{t,d}$ е броят на срещанията (или честота) на терм $t$ в документ $d$. Нека $idf(t,D) = \log \frac{|D|}{|\{d : d \in D \land t \in d \}|}$, където $D$ е корпуса, a $|\{d : d \in D \land t \in d \}|$ е броят на документите, където терма $t$ се среща. \textbf{TF-IDF} дефинираме като $tfidf(t,d,D) = tf(t,d) idf(t,D)$. Важно е да отбележим, че съществуват и други формули за дефиниране на \textbf{TF-IDF}, но дипломната работа разчита на тази.

\bigbreak

Класически пример за документ е приказка от сборник с такива, където термите са думи от естествен език. По-рядко срещан, но валиден пример за документ е изображение от множество с такива, където термите са точки на интерес от изображение. Такава интерпретация ще използваме в дипломната работа. Друго нетривиално, но важно за целите на този труд разбиране, е да мислим документ за множество от описания на изображение от множество с такива, където термите са описания.

\bigbreak

Детайлността на едно изображение се представя чрез неговите точки на интерес. \textbf{Точка на интерес} наричаме регион от изображението, който е устойчив на математически операции, защото е богат на информация, има ясно дефинирано разположение в пространството и е стабилен на локални и глобални трансформации в изображението. Нека имаме изображение $i_1$ с фотографично копие $i_2$ и точка на интерес $p_1$ в $i_1$. Лесно може да се определи, къде в $i_2$ се проектира $p_1$. Точките на интерес често обхващат значими за семантиката на едно изображение региони. Съответно те често са подходящи за изграждане на кодираща функция на изображения.

\bigbreak

Значим за тази дипломна работа метод за откриване на точки на интерес и изграждане на кодираща функция на изображения е \textbf{инвариантна към мащаби трансформация на атрибути} (\textbf{(Scale Invariant Feature Transform) SIFT} \cite{sift}). Те са инвариантни към скалиране, ротация, отместване и частично инвариантни към промяна на илюминация и локална дисторция. Откритите точки на интерес се явяват като апроксимация на тези, открити от главната визуална кора на приматите по това, че споделят подобно кодиране на форми и цвят \cite{primatevisualcortex}.

\bigbreak

При системите за извличане на информация съществуват две основни метрики за оценка на качеството - \textit{precision} и \textit{recall}. \textbf{\textit{Precision}} наричаме съотношението на броя на откритите релевантните елементи към броя на всички открити елементи. \textbf{\textit{Recall}} е съотношението на броя на откритите релевантни елементи към броя на всички релевантни елементи.

\bigbreak

\textbf{Онтология} наричаме структура, която представя същностите на дадена област, както и връзките между тях. В практиката онтология представяме чрез граф, където същностите и свойствата са върхове, а връзките между тях са ребра. Най-често срещаната форма на онтология е \textbf{домейн онтология}, която описва свойствата на дадена област от истинския свят. Примери за домейн онтологии са такива за домашни любимци, екология, грънчарство, археология и други. Същностите ще делим на концепти и индивиди. Концептите представляват множества от индивиди отговарящи на конкретни ограничения. Пример за концепт е `вино', а за индивид е `пино ноар от реколта 2005-та на чичо Минко'.

\bigbreak

Съществуват две дефиниции на \textbf{клас} - разширителна и ограничителна. Според разширителната, клас наричаме множество от индивиди (още нарачени обекти). Според ограничителната, клас наричаме абстрактни индивиди, ограничени по конкретни свойства. За дефиницията на \textbf{фасет} читателят може да реферира \cite{facets}.

% *************************************************************************************************************
\subsection{Обзор на съществуващите подходи за семантично търсене}

\subsubsection{Търсене чрез семантични техники и търсене на формално семантично анотирани документи}

Съдържанието на тази секция е до голяма степен заимствано от \cite{semanticsearchsurvey}. Подходите описани в тази секция зависят от наличието на коректно описани онтологии и използването им за правене на логически заключения. Знаем, че не съществува онтология, която коректно да описва целия свят. Съответно тези подходи не са достатъчно генерични. От друга страна знаем, че съществуват онтологии, които много добре описват конкретни домейни (напр. музейната индустрия). Следователно такива подходи могат добре да функционират на по-малък обхват. Проблем е, че повечето от тях разчитат на формално анотирани документи.

\bigbreak

Споменава се, че тези подходи изследват няколко основни направления. Това са аугментация на традиционно търсене по ключови думи чрез семантични техники, локиране на базов концепт, заявки с комплекси ограничения, решаване на проблем чрез логическо програмиране и откриване на свързващи пътища. Следва подробно описание на всяко от тях.

\bigbreak

Семантичното търсене чрез \textbf{аугментация на традиционното търсене по ключови думи чрез семантични техники} комбинира търсене по ключови думи и употреба на онтологични техники, по най-различни начини, с цел откриване и на формално семантично анотирани документи. Така се постига увеличение на \textit{precision} и \textit{recall}. Предимство на подходите от това направление е, че позволяват откриването на документи, които не са формално семантично анотирани.

\bigbreak

Някои от подходите от това направление разчитат на аугментация на заявките чрез обхождане на онтология от тип тезаурус. Този вид онтологии описват връзките между думите от даден език. Например кои думи са синоними и кои антоними. Известна тезаурус онтология е \textit{WordNet}. Обичайно тези подходи се свеждат до \textbf{(1)} откриване на ключовите думи в онтологията, \textbf{(2)} откриване на други концепти посредством обхождане на онтологията и \textbf{(3)} използване на откритите концепти за разширяване или ограничаване на заявката. Първата фаза обичайно включа традиционните стемиране, токенизация и премахване на стоп думи от заявката, последвани от асоцииране на думите с терми в онтологията. Във втората си фаза, тези подходи обичайно се различават в избора на алгоритъм за обхождане на онтология (напр. предимство при избор на някои свойства пред други), както и по избор на концепти. Примери за разширяване на заявката са \cite{wordnetinternetsearches} и \cite{wordnetgeo}, които разширяват множеството от ключови думи с това на откритите концепти посредством булевия оператор \textbf{`или'}. Пример за ограничаване на заявката се среща в \cite{cleversearch}, където това се извършва посредством булевия оператор \textbf{`и'}.

\bigbreak

Подходът, при който се \textbf{локира базов концепт}, разчита на разделяне на знанието на класове и обекти, където данните, които потребителят търси се определят като обекти, които принадлежат към конкретен клас. Данните са анотирани като такива. Познанието за домейна се представя като множество от класове с връзки между тях. Потребителят започва търсенето от частта на онтологията представяща домейн знанието, започвайки от най-абстрактния клас и стигайки до по конкретен такъв. Следва задаване на ограничения над свойствата на избрания клас. Знаейки това, системата може да върне обекти отговарящи на подадените ограничения. Примери за системи използващи този подход са \cite{shoe} и \cite{seal}.

\bigbreak

В най-базовата си форма този подход използва един фасет. Мощно негово разширение се получава чрез използването на множество фасети. Идеята тук е, че потребителят може да приложи различни ограничения над множество от класове. Търсене с няколко фасета се използва от \cite{ontoviews} и музеите на Финландия \cite{finlandmuseum}. В системата за търсене на музите на Финладния потребител задава различни свойства на обектите като материал на обекта, място на производство и контекст на употреба.

\bigbreak

Подходът за локиране на базов концепт се разширява и до търсене, при което потребителят не е на ясно със същността и свойствата на търсените обекти, но разполага с информация, чрез която може да започне търсене. Този процес на търсене се нарича \textbf{ориентиране} и според \cite{orienteering} е един от най-разпространените подходи при търсене. Системата предложена в \cite{haystack} имплементира локиране чрез базов концепт, но поддържа и търсене чрез връзка от един обект към друг свързан обект.

\bigbreak

complex constraint queries


% *************************************************************************************************************
\subsection{Избор на критерии за сравнение и сравнителен анализ на решения}

% *************************************************************************************************************
\subsection{Изводи}

% =============================================================================================================
\printbibliography[title={Използвана литература}]

\end{document}
